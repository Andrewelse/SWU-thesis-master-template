%%==================================================
%% chapter2.tex for SWU Master Thesis
%% modified by Zhang Lin
%% version: 1.0
%% last update: 20251217
%%==================================================
\chapter{一类xx模型}
\enchapter{A class of xx models}
\label{}
\section{模型建立}
\ensection{Modelling}

模型建立内容, 上标引用\supercite{Takahashi1996Structure,Xia2002Analysis,Jiang1989,Mao2000Motion,Feng1998}. 文献\cite{Mao2000Motion}中...

\section{基础结果}
\ensection{Basic results}
%\label{sec:***} 可标注label

% 1. 使用「定义」环境
\begin{defn}[函数连续性] % 可选:添加环境标题(括号内)
	设函数 $f(x)$ 在 $x_0$ 的某邻域内有定义,若
	\[
	\lim_{x \to x_0} f(x) = f(x_0),
	\]
	则称 $f(x)$ 在 $x_0$ 处连续。
\end{defn}

% 2. 使用「定理」环境(与定义共享编号,下一个编号是 1.2)
\begin{thm}[介值定理]
	若函数 $f(x)$ 在闭区间 $[a,b]$ 上连续,且 $f(a) \neq f(b)$,则对介于 $f(a)$ 和 $f(b)$ 之间的任意实数 $c$,存在 $\xi \in (a,b)$,使得 $f(\xi) = c$。
\end{thm}

\begin{proof}
%\begin{proof}[介值定理的证明] % 可选:添加证明标题
	不妨设 $f(a) < c < f(b)$,构造辅助函数 $g(x) = f(x) - c$,则 $g(x)$ 在 $[a,b]$ 上连续,且
	\[
	g(a) = f(a) - c < 0, \quad g(b) = f(b) - c > 0.
	\]
	由零点存在定理,存在 $\xi \in (a,b)$ 使得 $g(\xi) = 0$,即 $f(\xi) = c$。证毕。
\end{proof}

数学公式结尾的证明加'\text{\\qedhere}'调整.

\begin{proof}
	易证 $g(\xi) = 0$,即:
	\[
	f(\xi) = c \qedhere % QED 符号移到公式末尾
	\]
\end{proof}


% 3. 使用「例子」环境
\begin{exmp}
	$f(x) = x^2$ 在 $\mathbb{R}$ 上处处连续(验证定义 1.1)。
\end{exmp}

% 4. 使用「注记」环境
\begin{rem}
	连续函数一定有界吗?不一定(如 $f(x) = x$ 在 $\mathbb{R}$ 上连续但无界)。
\end{rem}



\section{稳定性分析和分支分析}
\ensection{Stability Analysis and Branch Analysis}

%\label{sec:features}


\subsection{形状记忆聚氨酯的研究进展}
%\label{sec:requirements}
首例SMPU是日本Mitsubishi公司开发成功的……。

\subsection{水系聚氨酯及聚氨酯整理剂}

由于它们对纤维织物的浸透性和亲和性不同,因此在纺织品染整加工中的用途也有差别,其中以水溶型和乳液型产品较为常用。另外,水系聚氨酯又有反应性和非反应性之分。虽然它们的共同特点是分子结构中不含异氰酸酯基,但前者是用封闭剂将异氰酸酯基暂时封闭,在纺织品整理时复出。相互交联反应形成三维网状结构而固着在织物表面。
……

然而索引扩散并不总是有效率的,它也会带来带宽开销。一方面,扩散更多的索引可以使搜索更快地返回,减少了搜索带宽开销;另一方面,由于P2P中结点和数据处于不断动态变化之中,当数据失效或更新时(如结点离线、删除或更新数据),数据的索引也相应失效,必须加以更新维护。因此,扩散更多的索引意味着维护开销的增加。于是在带宽开销方面,搜索开销与索引维护开销之间存在着折衷关系(trade-off)。与以往工作中仅考虑搜索开销不同,本章的模型中我们同时考虑搜索和维护两方面,给出了索引扩散方法对搜索整体性能的影响和数学关系。通过模型我们发现索引数量是决定宽松约束一般性搜索性能的至关重要的因素,采用最优索引分布可以很大程度上提高性能,降低系统开销。与一般认为的P2P无偏向性搜索难于扩展(non-scalable)恰恰相反,模型显示在最优的索引扩散策略下,基于无偏向性搜索具备很好的可扩展性,其结点负载和带宽开销随系统规模N(结点数)增长具有O的增长关系。这种平方根关系保证了对大规模P2P系统很好的适应性。

\gls{FFT}是离散傅立叶变换的一种快速算法。
\gls{IFFT}是离散傅立叶逆变换的一种快速算法。


